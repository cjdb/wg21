\rSec0[support]{Algorithm support}

\begin{quote}
``Generic Programming pro tip \#2: The "basis operations" of a well-designed concept or concept
hierarchy is the minimal set of operations that are both sufficient and necessary for efficiently
implementing all algorithms of interest within a particular domain.''
\begin{flushright}
\textemdash \textit{Eric Niebler, Twitter}
\end{flushright}
\end{quote}

The following subsections articulate the concept designs and any supporting material (such as
traits).

\reviewernote{
  This section's `wording' is not intended to be reviewed for wording (hence why the chapter isn't
  titled `Proposed Wording').
}

\rSec1[support.traits]{Numeric traits}

This section provides exposition for the traits that are used by algebraic concepts. The author is
aware that the design is not necessarily the most appropriate, and is open to suggestions for
improvement.

\rSec2[support.traits.identity]{Identity traits}

\pnum
An identity element $id$ is a special element in a set $S$, such that for all other elements $x$ in
$S$, given a magma $\cdot$, at least one of $x \cdot id = x$ or $id \cdot x = x$ holds.

\pnum
If both $x \cdot id = x$ and $id \cdot x = x$ hold, then $id$ is unique (\ref{proof.identity}).

\rSec3[support.traits.identity.left]{Left identity}

\pnum
\tcode{left_identity} is a type that represents the notion of a \textit{left-identity}.

\begin{itemdecl}
namespace std {
  template<class BOp, class T, class U = T>
  struct left_identity {};

  template<class BOp, class T, class U = T>
  using left_identity_t = decltype(declval<left_identity<BOp, T, U>>()());
}
\end{itemdecl}
\begin{itemdescr}
   \pnum
   A program may specialise \tcode{left_identity}, as described in the example below. No diagnostic is
   required for specialisations that do not follow this implementation. No diagnostic is required for
   explicit specialisations of the template parameters \tcode{T} or \tcode{U}.
\begin{example}
\begin{codeblock}
struct binary_op {
  int operator()(int, int) const;
};

namespace std {
  template<class T, class U>
  requires magma<binary_op, T, U>
  struct left_identity<binary_op, T, U> {
    constexpr common_type_t<T, U> operator()() const
    { /* implementation-defined */ }
  };
}
\end{codeblock}
\end{example}

\end{itemdescr}

\rSec3[support.traits.identity.right]{Right identity}

\pnum
\tcode{right_identity} is a type that represents the notion of a \textit{right-identity}.

\begin{itemdecl}
namespace std {
  template<class BOp, class T, class U = T>
  struct right_identity {};

  template<class BOp, class T, class U = T>
  using right_identity_t = decltype(declval<right_identity<BOp, T, U>>()());
}
\end{itemdecl}
\begin{itemdescr}
   \pnum
   A program may specialise \tcode{right_identity}, as described in the example below. No diagnostic
   is required for specialisations that do not follow this implementation. No diagnostic is required
   for explicit specialisations of the template parameters \tcode{T} or \tcode{U}.
\begin{example}
\begin{codeblock}
struct binary_op {
  int operator()(int, int) const;
};

namespace std {
  template<class T, class U>
  requires magma<binary_op, T, U>
  struct right_identity<binary_op, T, U> {
    constexpr common_type_t<T, U> operator()() const
    { /* implementation-defined */ }
  };
}
\end{codeblock}
\end{example}
\end{itemdescr}

\rSec3[support.traits.identity.two.sided]{Two-sided identity}

\pnum
\tcode{two_sided_identity} is a type that represents the notion of a \textit{two-sided identity}.

\begin{itemdecl}
namespace std {
  template<class BOp, class T, class U>
  concept @\placeholder{has-two-sided-identity}@ = // \expos
    requires(BOp bop, const T& t, const U& u) {
       typename left_identity_t<BOp, T, U>;
       typename left_identity_t<BOp, U, T>;
       typename right_identity_t<BOp, T, U>;
       typename right_identity_t<BOp, U, T>;

       requires same_as<left_identity_t<BOp, T, U>, left_identity_t<BOp, U, T>>;
       requires same_as<right_identity_t<BOp, T, U>, right_identity_t<BOp, U, T>>;
       requires same_as<left_identity_t<BOp, T, U>, right_identity_t<BOp, T, U>>;

    };
}
\end{itemdecl}
\begin{itemdescr}
   \pnum
   Let \tcode{left1} be an object of type \tcode{left_identity<BOp, T, U>}, \tcode{left2} be an object
   of type \tcode{left_identity<BOp, U, T>}, \tcode{right1} be an object of type
   \tcode{right_identity<BOp, T, U>}, and \tcode{right2} be an object of type
   \tcode{right_identity<BOp, U, T>}.

   \pnum
   The expressions \tcode{left1() == left2()}, \tcode{right1() == right2()}, and
   \tcode{left1() == right1()} are all \tcode{true}.

   \pnum
   If \tcode{t != left1()} is \tcode{true} and \tcode{u != right1()} is \tcode{true}, then the
   expressions \tcode{t == invoke(bop, t, right())} and \tcode{u == invoke(bop, left(), u)} are both
   \tcode{true}.
\end{itemdescr}

\begin{itemdecl}
  template<class BOp, class T, class U = T>
  requires @\placeholder{has-two-sided-identity}@<BOp, T, U>
  struct two_sided_identity {
     constexpr common_type_t<T, U> operator()() const;
  };

  template<class BOp, class T, class U>
  using two_sided_identity_t =
    decltype(two_sided_identity{}(declval<BOp&>(), declval<T>(), declval<U>()));
\end{itemdecl}

\begin{itemdecl}
constexpr common_type_t<T, U> operator()(BOp bop, T&& t, U&& u) const;
\end{itemdecl}
\begin{itemdescr}
   \pnum
   \expects \tcode{left_identity<BOp, T, U>{}() == right_identity<BOp, T, U>()} is \tcode{true}.

   \pnum
   \effects Equivalent to:
   \begin{codeblock}
      return left_identity<BOp, T, U>{}();
   \end{codeblock}
\end{itemdescr}

\rSec2[support.traits.zero]{Zero traits}

\pnum
A zero element $z$ is a special element in a set $S$, such that for all other elements $x$ in $S$,
given a magma $\cdot$, at least one of $x \cdot z = z$ or $z \cdot x = z$ holds.

\pnum
If both $x \cdot z = z$ and $z \cdot x = z$ hold, then $z$ is unique (\ref{proof.zero}).

\rSec3[support.traits.zero.left]{Left zero}

\pnum
\tcode{left_zero} is a type that represents the notion of a \textit{left-zero}.

\begin{itemdecl}
namespace std {
  template<class BOp, class T, class U = T>
  struct left_zero {};

  template<class BOp, class T, class U = T>
  using left_zero_t = decltype(declval<left_zero<BOp, T, U>>()());
}
\end{itemdecl}
\begin{itemdescr}
   \pnum
   A program may specialise \tcode{left_zero}, as described in the example below. No diagnostic is
   required for specialisations that do not follow this implementation. No diagnostic is required for
   explicit specialisations of the template parameters \tcode{T} or \tcode{U}.
\begin{example}
\begin{codeblock}
struct binary_op {
  int operator()(int, int) const;
};

namespace std {
  template<class T, class U>
  requires magma<binary_op, T, U>
  struct left_zero<binary_op, T, U> {
    constexpr common_type_t<T, U> operator()() const
    { /* implementation-defined */ }
  };
}
\end{codeblock}
\end{example}

\end{itemdescr}

\rSec3[support.traits.zero.right]{Right zero}

\pnum
\tcode{right_zero} is a type that represents the notion of a \textit{right-zero}.

\begin{itemdecl}
namespace std {
  template<class BOp, class T, class U = T>
  struct right_zero {};

  template<class BOp, class T, class U = T>
  using right_zero_t = decltype(declval<right_zero<BOp, T, U>>()());
}
\end{itemdecl}
\begin{itemdescr}
   \pnum
   A program may specialise \tcode{right_zero}, as described in the example below. No diagnostic
   is required for specialisations that do not follow this implementation. No diagnostic is required
   for explicit specialisations of the template parameters \tcode{T} or \tcode{U}.
\begin{example}
\begin{codeblock}
struct binary_op {
  int operator()(int, int) const;
};

namespace std {
  template<class T, class U>
  requires magma<binary_op, T, U>
  struct right_zero<binary_op, T, U> {
    constexpr common_type_t<T, U> operator()() const
    { /* implementation-defined */ }
  };
}
\end{codeblock}
\end{example}
\end{itemdescr}

\rSec3[support.traits.zero.two.sided]{Two-sided zero}

\pnum
\tcode{two_sided_zero} is a type that represents the notion of a \textit{two-sided zero}.

\begin{itemdecl}
namespace std {
  template<class BOp, class T, class U>
  concept @\placeholder{has-two-sided-zero}@ = // \expos
    requires(BOp bop, const T& t, const U& u) {
       typename left_zero_t<BOp, T, U>;
       typename left_zero_t<BOp, U, T>;
       typename right_zero_t<BOp, T, U>;
       typename right_zero_t<BOp, U, T>;

       requires same_as<left_zero_t<BOp, T, U>, left_zero_t<BOp, U, T>>;
       requires same_as<right_zero_t<BOp, T, U>, right_zero_t<BOp, U, T>>;
       requires same_as<left_zero_t<BOp, T, U>, right_zero_t<BOp, T, U>>;
    };
}
\end{itemdecl}
\begin{itemdescr}
   \pnum
   Let \tcode{left1} be an object of type \tcode{left_zero<BOp, T, U>}, \tcode{left2} be an object of
   type \tcode{left_zero<BOp, U, T>}, \tcode{right1} be an object of type
   \tcode{right_zero<BOp, T, U>}, and \tcode{right2} be an object of type
   \tcode{right_zero<BOp, U, T>}.

   \pnum
   The expressions \tcode{left1() == left2()}, \tcode{right1() == right2()}, and
   \tcode{left1() == right1()} are all \tcode{true}.

   \pnum
   If \tcode{t != left1()} is \tcode{true} and \tcode{u != right1()} is \tcode{true}, then the
   expressions \tcode{right1() == invoke(bop, t, right1())} and
   \tcode{left1() == invoke(bop, left1(), u)} are both \tcode{true}.
\end{itemdescr}

\begin{itemdecl}
  template<class BOp, class T, class U>
  requires @\placeholder{has-two-sided-zero}@<BOp, T, U>
  struct two_sided_zero {
     constexpr common_type_t<T, U> operator()(BOp bop, T&& t, U&& u) const;
  };

  template<class BOp, class T, class U = T>
  using two_sided_zero_t =
    decltype(two_sided_zero{}(declval<BOp&>(), declval<T>(), declval<U>()));
\end{itemdecl}

\begin{itemdecl}
constexpr common_type_t<T, U> operator()(BOp bop, T&& t, U&& u) const;
\end{itemdecl}
\begin{itemdescr}
   \pnum
   \expects \tcode{left_zero<BOp, T, U>{}() == right_zero<BOp, T, U>{}()} is \tcode{true}.

   \pnum
   \effects Equivalent to:
   \begin{codeblock}
      return left_zero<BOp, T, U>();}
   \end{codeblock}
\end{itemdescr}

\rSec2[support.traits.inverse]{Inverse traits}

\reviewernote{
  The design for \tcode{inverse_traits} needs to be thoroughly revised, as it is overly restrictive,
  incomplete, and wrong.
}

\pnum
\tcode{inverse_traits} denotes a type that takes an object modelling a \tcode{magma} over an
arbitrary domain as input, and returns an object whose type models a \tcode{magma} over the same
domain, where the returned object is the mathematical inverse operation of the input.

\begin{itemdecl}
  template<class BOp>
  struct inverse_traits {};

  template<class BOp>
  using inverse_operation_t = typename inverse_traits::type;
\end{itemdecl}
\begin{itemdescr}
   \pnum
   Users may specialise this type, provided it is equivalent to:
   \begin{codeblock}
struct plus;
struct minus;

namespace std {
  template<>
  struct inverse_traits<plus> {
     using type = minus;
     constexpr type operator()() const { return type{}; }
  };

  template<>
  struct inverse_traits<minus> {
     using type = plus;
     constexpr type operator()() const { return type{}; }
  };
}
   \end{codeblock}

   \pnum
   Let \tcode{x} be an object of type \tcode{T}, \tcode{y} be an object of type \tcode{U},
   \tcode{bop} be an object of type \tcode{BOp}, where \tcode{BOp} models \tcode{magma<T, U>}, and
   \tcode{inv} be an object of type \tcode{Inv}, where \tcode{Inv} models \tcode{magma<T, U>}.

   \pnum
   \mandates
      \begin{itemize}
         \item If \tcode{is_same_v<inverse_type_t<BOp>, Inv>} is \tcode{true}, then
               \tcode{is_same_v<inverse_type_t<Inv>, BOp>} is also \tcode{true}, no diagnostic
               required.
         \item \tcode{is_same_v<inverse_type_t<BOp>, BOp>} is \tcode{false}, no diagnostic required.
      \end{itemize}
   \pnum
   \expects
      \begin{itemize}
         \item \tcode{invoke(inv, invoke(bop, x, y), y) == x} is \tcode{true}.
         \item \tcode{invoke(inv, invoke(bop, x, y), x) == y} is \tcode{true}.
      \end{itemize}
\end{itemdescr}

\reviewernote{
   This type could be relaxed to permit the inverse of other operations. For example, the inverse of
   $-x$ is $-(-x)$, and the inverse of \tcode{swap(x, y)} is \tcode{swap(x, y)}.
}

\rSec1[support.concepts]{Algebraic concepts}

This section describes the concepts required by the numeric algorithms, and are not present in the
\Cpp{}20 standard library.

\rSec2[support.concepts.commutative.op]{Concept commutative operation}

\pnum
A commutative operation is a binary operation where the order of its operands does not change the
evaluation of the expression.

\pnum
\begin{example}
   Integral arithmetic is commutative.
\end{example}

\pnum
\begin{example}
   Matrix multiplication is not commutative.
\end{example}

\begin{itemdecl}
  template<class BOp, class T, class U>
  concept commutative_operation =
    regular_invocable<BOp, T, U> &&
    regular_invocable<BOp, U, T> &&
    common_with<T, U> &&
    equality_comparable_with<T, U>;
\end{itemdecl}
\begin{itemdescr}
   \pnum
   Let \tcode{bop} be an object of type \tcode{BOp}, \tcode{t} be an object of type \tcode{T}, and
   \tcode{u} be an object of type \tcode{U}, where \tcode{t != u}.

   \pnum
   The result of \tcode{invoke(bop, t, u)} is expression-equivalent to \tcode{invoke(bop, u, t)}.
\end{itemdescr}

\rSec2[support.concepts.magma]{Concept magma}

\pnum
A \textit{magma} is a set $S$ associated with a binary operation $\cdot$, such that $S$ is
\textit{closed under} $\cdot$.

\pnum
\begin{example}
   $(\mathbb{Z}, +)$ is a magma, since we can add any two integers and find that the result is also
   an integer.
\end{example}

\pnum
\begin{example}
   $(\mathbb{Z}, /)$ is not a magma, since $\frac{2}{3}$ is not an integer.
\end{example}

\reviewernote{
   The term \textit{semigroupoid} is an older name for the notion of a magma, but this seems to have
   been co-opted by category theory.
}

\begin{itemdecl}
  template<class BOp, class T, class U>
  concept magma =
    common_with<T, U> &&
    regular_invocable<BOp, T, T> &&
    regular_invocable<BOp, U, U> &&
    regular_invocable<BOp, T, U> &&
    regular_invocable<BOp, U, T> &&
    common_with<invoke_result_t<BOp&, T, U>, T> &&
    common_with<invoke_result_t<BOp&, T, U>, U> &&
    same_as<invoke_result_t<BOp&, T, U>, invoke_result_t<BOp&, U, T>>;
\end{itemdecl}
\begin{itemdescr}
   \pnum
   Let \tcode{bop} be an object of type \tcode{BOp}, \tcode{t} be an object of type \tcode{T}, and
   \tcode{u} be an object of type \tcode{U}.

   \pnum
   The value \tcode{invoke(bop, t, u)} must return a result that is representable by
   \tcode{common_type_t<T, U>}.
\end{itemdescr}

The decision to require common types for a over \tcode{magma<T, U>} is similar to the reason that
\tcode{equality\_comparable\_with} requires \tcode{common\_reference\_with}: this ensures that when
an algorithm requires a \tcode{magma}, we are able to \textit{equationally reason} about those
requirements. It's possible to overload \tcode{operator+(int, vector<int> const\&)}, but that
doesn't follow the canonical usage of $+$. Does \tcode{1 + vector\{1, 2, 3\}} mean `concatenate
\tcode{vector\{1, 2, 3\}} to the end of a temporary \tcode{vector\{1\}}'? Is it a shorthand for
\tcode{accumulate(vector\{1, 2, 3\}, 1)}? The intention is unclear, and so \tcode{std::plus<>} (sic)
should not model \tcode{magma<int, vector<int{>}>}.

\reviewernote{
  This implies that the author is not a fan of $+$ as a way of concatenating strings. It's not
  ideal, but since this is already a standard practice, the author has chosen not to --- probably
  fruitlessly --- wage war on this specific case.
}

Similarly, if the result type is not related to its parameters, the operation lacks the ability to
be equationally reasoned about. While a programmer might have a good local reason to write a matrix
multiplication declared as \tcode{mat16 operator*(mat4 const\& x, mat4 const\&)}, this does not
adhere to the usual rules of mathematics, and is out-of-scope for generic programming.

\rSec2[support.concepts.semigroup]{Concept semigroup}

\pnum
A \textit{semigroup} $(S, \cdot)$ refines the concept of a magma, by requiring $\cdot$ to be an
\textit{associative} binary operation.

\pnum
\begin{example}
   $(\mathbb{R}, +)$ is a semigroup, since $(1.2 + 2.3) + \pi = 1.2 + (2.3 + \pi)$.
\end{example}

\pnum
\begin{example}
   $(\mathbb{R}, -)$ is not a semigroup, since $(1.2 - 2.3) - \pi \neq 1.2 - (2.3 - \pi)$.
\end{example}

\begin{itemdecl}
  template<class BOp, class T, class U>
  concept semigroup = magma<BOp, T, U>;
\end{itemdecl}
\begin{itemdescr}
   \pnum
   Let \tcode{bop} be an object of type \tcode{BOp}, \tcode{t} be an object of type \tcode{T}, and
   \tcode{u} be an object of type \tcode{U}.

   \pnum
   \tcode{invoke(bop, t, invoke(bop, t, u))} is expression-equivalent to
   \tcode{invoke(bop, invoke(bop, t, u), u)}.

   \pnum
   \begin{note}
      The difference between \tcode{magma} and \tcode{semigroup} is purely semantic.
   \end{note}
\end{itemdescr}

\rSec2[support.concepts.monoid]{Concept monoid}

\pnum
A \textit{monoid} $(S, \cdot)$ refines the concept of a semigroup by requiring $\cdot$ to have a
two-sided identity element.

\begin{itemdecl}
  template<class BOp, class T, class U = T>
  concept monoid = semigroup<BOp, T, U> && requires {
    typename two_sided_identity_t<BOp, remove_cvref_t<T>, remove_cvref_t<U>>;
  };
\end{itemdecl}

\rSec2[support.concepts.quasigroup]{Concept quasigroup}

\pnum
A \textit{quasigroup} $(S, \cdot)$ refines the concept of a magma by requiring $\cdot$ have an
inverse operation.

\begin{example}
   $(\mathbb{Z}, +)$ is a quasigroup, since subtraction is the inverse operation of addition.
\end{example}

\begin{example}
   $(\mathbb{Z}, \text{rem})$ is not a quasigroup, since there is no inverse operation for remainder.
\end{example}

\begin{itemdecl}
  template<class BOp, class T, class U>
  concept quasigroup = magma<BOp, T, U> and requires {
     typename inverse_operation_t<BOp, T, U>;
     typename inverse_operation_t<inverse_operation_t<BOp, T, U>, T, U>;
     requires same_as<BOp, inverse_operation_t<inverse_operation_t<BOp, T, U>, T, U>>;
  };
\end{itemdecl}
\begin{itemdescr}
   \pnum
   Let \tcode{t} be an object of type \tcode{T}, \tcode{u} be an object of type \tcode{U},
   \tcode{bop} be an object of type \tcode{BOp}.

   \pnum
   There exists an object \tcode{inv} of type \tcode{Inv}, where \tcode{Inv} models
   \tcode{magma<T, U>}, and the expressions
   \begin{itemize}
      \item \tcode{invoke(inv, invoke(bop, t, u), t) == u}
      \item \tcode{invoke(inv, invoke(bop, t, u), u) == t}
   \end{itemize}
   are both \tcode{true}.
   \begin{note}
      This implies that \tcode{Inv} also models \tcode{quasigroup<T, U>}, with respect to
      \tcode{inv}.
   \end{note}
\end{itemdescr}

\reviewernote{
   The author wonders if having both \tcode{semigroup} and \tcode{quasigroup} will be confusing for
   some people. The names \tcode{associative_magma} (for \tcode{semigroup}) and
   \tcode{invertible_magma} (for \tcode{quasigroup}) have been considered as alternatives.
}

\rSec2[support.concepts.loop]{Concept loop}

\pnum
A \textit{loop} $(S, \cdot)$ refines the concept of a quasigroup by requiring $\cdot$ to have a
two-sided identity element.

\begin{itemdecl}
  template<class BOp, class T, class U>
  concept loop = quasigroup<BOp, T, U> && requires {
    typename two_sided_identity_t<BOp, remove_cvref_t<T>, remove_cvref_t<U>>;
  };
\end{itemdecl}
\begin{itemdescr}
   \pnum
   \tcode{invoke(inv, t, invoke(bop, t, u))} is expression-equivalent to
   \tcode{invoke(inv, two_sided_identity{}(bop, t, u), t)}.

   \pnum
   \tcode{invoke(inv, u, invoke(bop, t, u))} is expression-equivalent to
   \tcode{invoke(inv, two_sided_identity{}(bop, t, u), u)}.
\end{itemdescr}

\reviewernote{
   To avoid confusion between computer science loops and abstract algebra loops, the author is
   considering renaming \tcode{loop} to \tcode{abstract\_loop} or \tcode{algebraic\_loop}.\\
   \\
   The solution should be considered for the notion of a module (for obvious reasons), ring (to
   avoid confusion with types such as \tcode{ring\_buffer}), and to a lesser extent, group (`group'
   is a \textit{very} abstract term). This proposal introduces neither a \tcode{module} concept, nor
   a \tcode{ring} concept (but it is plausible that a subsequent paper propose these).
}

\rSec2[support.concepts.group]{Concept group}

\pnum
A \textit{group} is a refinement of both a semigroup and a quasigroup\cite{wikipedia_group}.

\begin{itemdecl}
  template<class BOp, class T, class U>
  concept group = semigroup<BOp, T, U> && quasigroup<BOp, T, U>;
\end{itemdecl}

\reviewernote{
  The author is open to simplifying the group heirarchy by following the design presented in EoP,
  where a group refines a monoid by requiring inverse operation. This would eliminate the confusion
  between `semigroup' and `quasigroup', and won't introduce any concept-like diamond problems.
}

\rSec2[support.concepts.abelian.group]{Concept abelian group}

\pnum
A \textit{abelian group} refines a group such that the operation is
commutative\cite{wikipedia_abelian}.

\begin{itemdecl}
  template<class BOp, class T, class U>
  concept abelian_group = group<BOp, T, U> && commutative_operation<BOp, T, U>;
\end{itemdecl}

\reviewernote{
   An alternative, and possibly more appropriate name for this concept is \tcode{commutative_group}.
}

\reviewernote{
   \tcode{group} and \tcode{abelian_group} are the only concepts introduced in this proposal that
   aren't required by an algorithm, but given their `simple' definitions, it would be remiss to omit
   them. The same cannot be said for the ring hierarchy, as there are more gaps between what is
   proposed in P1813 and the full definition of a ring.
}

\rSec2[support.concepts.weak.magmaring]{Concept weak-magmaring}

\pnum
A \textit{weak-magmaring} $(S, \cdot)$ is a generalisation of the notion of a near-semiring, where:
\begin{itemize}
   \item $(S, +)$ is a magma.
   \item $(S, \cdot)$ is a magma.
   \item $\cdot$ is distributive over $+$
   \begin{itemize}
      \item $a \cdot (b + c) = (a \cdot b) + (a \cdot c)$
      \item $(a + b) \cdot c = (a \cdot c) + (b \cdot c)$
   \end{itemize}
\end{itemize}
\begin{note}
   In the definition of a weak-magmaring, $+$ does not refer to canonical addition, and $\cdot$ does
   not refer to canonical multiplication.
\end{note}

\begin{itemdecl}
  template<class BOp1, class BOp2, class T, class U, class V>
  concept weak_magmaring = magma<BOp2, U, V> && magma<BOp1, T, invoke_result_t<BOp2&, U, V>>;
\end{itemdecl}
\begin{itemdescr}
   \pnum
   Let \tcode{bop1} be an object of type \tcode{BOp1}, \tcode{bop2} be an object of type
   \tcode{BOp2}, \tcode{t} be an object of type \tcode{T}, \tcode{u} be an object of type \tcode{U},
   and \tcode{v} be an object of type \tcode{V}.

   \pnum
   \tcode{invoke(bop2, invoke(bop1, t, u), v)} is expression-equivalent to
   \tcode{invoke(bop1, invoke(bop2, t, v), invoke(bop2, u, v))}.
\end{itemdescr}

\reviewernote{
   This could be renamed as \tcode{distributive_operation} to avoid introducing novel mathematical
   terms that lack rigourous definitions. The author is not convinced that the concept definition
   needs to change for this renaming to be possible.
}

\rSec2[support.concepts.near.semiring]{Concept near-semiring}

\pnum
A \textit{near-semiring} $(S, +, \cdot)$ refines the notion of a weak-magmaring, by refining the
substructures, and introducing the notion of a two-sided zero element.
\begin{itemize}
   \item $(S, +)$ is a monoid.
   \item $(S, \cdot)$ is a semigroup.
   \item $0 \cdot a = a \cdot 0 = 0$ for all $a$ in $S$.
\end{itemize}

\begin{itemdecl}
  template<class BOp1, class BOp2, class T, class U, class V>
  concept near_semiring = weak_magmaring<BOp1, BOp2, T, U, V> &&
    monoid<BOp1, T, invoke_result_t<BOp2&, U, V>> && semigroup<BOp2, U, V> && requires {
      typename two_sided_zero_t<BOp2, remove_cvref_t<U>, remove_cvref_t<V>>;
    };
\end{itemdecl}

\rSec2[support.concepts.indirect]{Indirect callable requirements}

The following concepts are convenience concepts, similar to those already in the \Cpp{}20 WP. With
the exception of \tcode{indirect_commutative_operation}, all of the proposed concepts in this
section require the algebraic structure model \tcode{writable} to some object. This makes the
indirect algebraic structure concepts more in line with \tcode{sortable} and \tcode{permutable} than
with \tcode{indirect_unary_invocable}, etc.

\begin{itemdecl}
template<class BOp, class I1, class I2>
concept indirect_commutative_operation =
  readable<I1> &&
  readable<I2> &&
  commutative_operation<BOp&, iter_value_t<I1>&, iter_value_t<I2>&> &&
  commutative_operation<BOp&, iter_value_t<I1>&, iter_reference_t<I2>> &&
  commutative_operation<BOp&, iter_reference_t<I1>, iter_value_t<I2>&> &&
  commutative_operation<BOp&, iter_reference_t<I1>, iter_reference_t<I2>> &&
  commutative_operation<BOp&, iter_common_reference_t<I1>, iter_common_reference_t<I2>>;

template<class BOp, class I1, class I2, class O>
concept indirect_magma =
  readable<I1> &&
  readable<I2> &&
  writable<O, indirect_result_t<BOp&, I1, I2>> &&
  magma<BOp&, iter_value_t<I1>&, iter_value_t<I2>&> &&
  magma<BOp&, iter_value_t<I1>&, iter_reference_t<I2>&> &&
  magma<BOp&, iter_reference_t<I1>, iter_value_t<I2>&> &&
  magma<BOp&, iter_reference_t<I1>, iter_reference_t<I2>> &&
  magma<BOp&, iter_common_reference_t<I1>, iter_common_reference_t<I2>>;

template<class BOp, class I1, class I2, class O>
concept indirect_semigroup = indirect_magma<BOp, I1, I2, O> &&
  semigroup<BOp&, iter_value_t<I1>&, iter_value_t<I2>&> &&
  semigroup<BOp&, iter_value_t<I1>&, iter_reference_t<I2>&> &&
  semigroup<BOp&, iter_reference_t<I1>, iter_value_t<I2>&> &&
  semigroup<BOp&, iter_reference_t<I1>, iter_reference_t<I2>> &&
  semigroup<BOp&, iter_common_reference_t<I1>, iter_common_reference_t<I2>>;

template<class BOp, class I1, class I2, class O>
concept indirect_monoid = indirect_semigroup<BOp, I1, I2, O> &&
  monoid<BOp&, iter_value_t<I1>&, iter_value_t<I2>&> &&
  monoid<BOp&, iter_value_t<I1>&, iter_reference_t<I2>&> &&
  monoid<BOp&, iter_reference_t<I1>, iter_value_t<I2>&> &&
  monoid<BOp&, iter_reference_t<I1>, iter_reference_t<I2>> &&
  monoid<BOp&, iter_common_reference_t<I1>, iter_common_reference_t<I2>>;

template<class BOp1, class BOp2, class I1, class I2, class I3, class O>
concept indirect_weak_magmaring =
  indirect_magma<BOp2, I2, I3, O> &&
  indirect_magma<BOp1, I1, indirect_result_t<BOp2&, I2, I3>*, O> &&
  weak_magmaring<BOp1&, BOp2&, iter_value_t<I1>&, iter_value_t<I2>&, iter_value_t<I3>&> &&
  weak_magmaring<BOp1&, BOp2&, iter_value_t<I1>&, iter_value_t<I2>&, iter_reference_t<I3>> &&
  weak_magmaring<BOp1&, BOp2&, iter_value_t<I1>&, iter_reference_t<I2>, iter_value_t<I3>&> &&
  weak_magmaring<BOp1&, BOp2&, iter_value_t<I1>&, iter_reference_t<I2>, iter_reference_t<I3>> &&
  weak_magmaring<BOp1&, BOp2&, iter_reference_t<I1>, iter_value_t<I2>&, iter_value_t<I3>&> &&
  weak_magmaring<BOp1&, BOp2&, iter_reference_t<I1>, iter_value_t<I2>&, iter_reference_t<I3>> &&
  weak_magmaring<BOp1&, BOp2&, iter_reference_t<I1>, iter_reference_t<I2>, iter_value_t<I3>&> &&
  weak_magmaring<BOp1&, BOp2&, iter_reference_t<I1>, iter_reference_t<I2>, iter_reference_t<I3>> &&
  weak_magmaring<BOp1&, BOp2&, iter_common_reference_t<I1>, iter_common_reference_t<I2>,
    iter_common_reference_t<I3>>;

template<class BOp1, class BOp2, class I1, class I2, class I3, class O>
concept indirect_near_semiring =
  indirect_weak_magmaring<BOp1, BOp2, I1, I2, I3, O> &&
  near_semiring<BOp1&, BOp2&, iter_value_t<I1>&, iter_value_t<I2>&, iter_value_t<I3>&> &&
  near_semiring<BOp1&, BOp2&, iter_value_t<I1>&, iter_value_t<I2>&, iter_reference_t<I3>> &&
  near_semiring<BOp1&, BOp2&, iter_value_t<I1>&, iter_reference_t<I2>, iter_value_t<I3>&> &&
  near_semiring<BOp1&, BOp2&, iter_value_t<I1>&, iter_reference_t<I2>, iter_reference_t<I3>> &&
  near_semiring<BOp1&, BOp2&, iter_reference_t<I1>, iter_value_t<I2>&, iter_value_t<I3>&> &&
  near_semiring<BOp1&, BOp2&, iter_reference_t<I1>, iter_value_t<I2>&, iter_reference_t<I3>> &&
  near_semiring<BOp1&, BOp2&, iter_reference_t<I1>, iter_reference_t<I2>, iter_value_t<I3>&> &&
  near_semiring<BOp1&, BOp2&, iter_reference_t<I1>, iter_reference_t<I2>, iter_reference_t<I3>> &&
  near_semiring<BOp1&, BOp2&, iter_common_reference_t<I1>, iter_common_reference_t<I2>,
    iter_common_reference_t<I3>>;
\end{itemdecl}

\rSec1[support.arithmetic.ops]{Arithmetic function objects}

Just as P0896 redesigned the comparison function objects, P1813 seeks to redesign the numeric
operation function objects. This will allow us to:

\begin{itemize}
   \item Forget about --- and hopefully --- one day eliminate the arithmetic function objects in
         namespace \tcode{std}.
   \item Apply requirements to each operation to ensure that they're semantically sound (what does
         it `mean' to evaluate \tcode{1 + vector\{1, 2, 3\}}?).
\end{itemize}

\rSec2[support.arithmetic.plus]{Plus}

\begin{itemdecl}
namespace std::ranges {
  template<class T, class U>
  concept @\placeholder{summable-with}@ = // \expos
    default_initializable<remove_reference_t<T>> &&
    default_initializable<remove_reference_t<U>> &&
    common_reference_with<T, U> &&
    requires(T&& t, U&& u) {
      { std::forward<T>(t) + std::forward<T>(t) } -> common_with<T>;
      { std::forward<U>(u) + std::forward<U>(u) } -> common_with<U>;
      { std::forward<T>(t) + std::forward<U>(u) } -> common_with<T>;
      { std::forward<U>(u) + std::forward<T>(t) } -> common_with<U>;
      requires same_as<decltype(std::forward<T>(t) + std::forward<U>(u)),
                       decltype(std::forward<U>(u) + std::forward<T>(t))>;
    };
}
\end{itemdecl}
\begin{itemdescr}
  \pnum
  The expression \tcode{t + u} is expression-equivalent to \tcode{u + t}.

  \pnum
  The expressions \tcode{t + T\{\} == t}, \tcode{u + T\{\} == common_type_t<T, U>\{\}}, and
  \tcode{t + U\{\} == common\_type\_t<T, U>\{\}} are all \tcode{true}.
\end{itemdescr}
\reviewernote{
   This is not the same as the dreaded \tcode{has_plus}; it's more of a \tcode{has_plus+}.
}

\begin{itemdecl}
  struct plus {
    template<class T, @\placeholder{summable-with}@<T> U>
    constexpr decltype(auto) operator()(T&& t, U&& u) const {
      return std::forward<T>(t) + std::forward<U>(u);
    }

    using is_transparent = std::true_type;
  };

  template<class T, class U>
  requires magma<ranges::plus, T, U>
  struct left_identity<ranges::plus> {
    constexpr common_type_t<T, U> operator()() const { return T{}; }
  };

  template<class T, class U>
  requires magma<ranges::plus, T, U>
  struct right_identity<ranges::plus> {
    constexpr common_type_t<T, U> operator()() const { return U{}; }
  };

  template<>
  struct inverse_traits<ranges::plus> {
    using type = minus;
    constexpr type operator()() const noexcept { return type{}; }
  };
}
\end{itemdecl}

\rSec2[support.arithmetic.negate]{Negate}

\begin{itemdecl}
namespace std::ranges {
  template<class T>
  concept @\placeholder{negatable}@ = // \expos
    @\placeholder{summable-with}@<T, T> &&
    totally_ordered<T> &&
    requires(T&& t) {
      { -std::forward<T>(t) } -> common_with<T>;
    };
\end{itemdecl}
\begin{itemdescr}
   \pnum
   Let \tcode{t}, \tcode{t1}, and \tcode{t2} objects of type \tcode{T}.

   \pnum
   \tcode{-(-t)} is expression-equivalent to \tcode{t}.

   \pnum
   The expression \tcode{-t == t} is \tcode{true} if, and only if, \tcode{t == T\{\}} is also
   \tcode{true}.

   \pnum
   The expression \tcode{-t < t} is \tcode{true} if, and only if, \tcode{T\{\} < t} is also
   \tcode{true}.

   \pnum
   The expression \tcode{t + -t} is expression-equivalent to \tcode{T\{\}}.

   \pnum
   If \tcode{t1 < t2} is \tcode{true} and \tcode{T\{\} < t2} is \tcode{true}, then
   \begin{itemize}
      \item \tcode{t1 + -t2 < t1} is \tcode{true},
      \item \tcode{t1 + -t2 > -t2} is \tcode{true},
      \item \tcode{-t1 + t2 < t2} is \tcode{true}, and
      \item \tcode{-t1 + t2 > t1} is \tcode{true}.
   \end{itemize}
\end{itemdescr}

\begin{itemdecl}
  struct negate {
    template<@\placeholder{negatable}@ T>
    constexpr decltype(auto) operator()(T&& t) const {
      return -std::forward<T>(t);
    }

    using is_transparent = std::true_type;
  };
}
\end{itemdecl}

\reviewernote{
   \tcode{negate} is not a binary operation, but it is plausible for there to be an
   \tcode{inverse_operation<negate>} specialisation where \tcode{operator()} returns an object of
   type \tcode{negate}.
}

\rSec2[support.arithmetic.minus]{Minus}

\begin{itemdecl}
namespace std::ranges {
  template<class T, class U>
  concept @\placeholder{differenceable-with}@ = // \expos
    @\placeholder{summable-with}@<T, U> &&
    @\placeholder{negatable}@<T> &&
    @\placeholder{negatable}@<U> &&
    totally_ordered_with<T, U> &&
    requires(T&& t, U&& u) {
      { std::forward<T>(t) - std::forward<T>(t) } -> common_with<T>;
      { std::forward<U>(u) - std::forward<U>(u) } -> common_with<U>;
      { std::forward<T>(t) - std::forward<U>(u) } -> common_with<T>;
      { std::forward<U>(u) - std::forward<T>(t) } -> common_with<U>;
      requires same_as<decltype(std::forward<T>(t) - std::forward<U>(u)),
                       decltype(std::forward<U>(u) - std::forward<T>(t))>;
    };
\end{itemdecl}
\begin{itemdescr}
   \pnum
   Let \tcode{t1} and \tcode{t2} be objects of type \tcode{T}, and \tcode{u1} and \tcode{u2} be
   objects of type \tcode{U}, where \tcode{t1 != t2} and \tcode{u1 != u2}.

   \pnum
   \tcode{t1 - t2} is equivalent to \tcode{t1 + -t2}, \tcode{u1 - u2} is equivalent to
   \tcode{u1 + -u2}, and \tcode{t - u} is equivalent to \tcode{t + -u}.

   \pnum
   \tcode{t - t} is expression-equivalent to \tcode{T\{\}}, \tcode{u - u} is expression-equivalent
   to \tcode{U\{\}}, and if \tcode{t == u}, then \tcode{t - u} is expression-equivalent to
   \tcode{common_type_t<T, U>\{\}}.

   \pnum
   \tcode{t - (-t)} is equivalent to \tcode{t + t}, \tcode{u - (-u)} is equivalent to \tcode{u + u},
   and \tcode{t - (-u)} is equivalent to \tcode{t + u}.

   \pnum
   \tcode{-t1 - t2} is equivalent to \tcode{-(t1 + t2)}, \tcode{-u1 - u2} is equivalent to
   \tcode{-(u1 + u2)}, and \tcode{-t - u} is expression-equivalent to \tcode{-(t + u)}.

   \pnum
   \tcode{t + u - t} is expression-equivalent to \tcode{static_cast<common_type_t<T, U>(t)}, and
   \tcode{t + u - u} is expression-equivalent to \tcode{static_cast<common_type_t<T, U>(u)}.
\end{itemdescr}

\reviewernote{
  TODO: add semantics for subtraction and ordering.
}

\begin{itemdecl}
  struct minus {
    template<class T, @\placeholder{differenceable-with}@<T> U>
    constexpr decltype(auto) operator()(T&& t, U&& u) const {
      return std::forward<T>(t) - std::forward<U>(u);
    }

    using is_transparent = std::true_type;
  };

  template<class T, class U>
  struct right_identity<ranges::minus, T, U> : private right_identity<ranges::plus, T, U> {
    using right_identity<ranges::plus, T, U>::operator();
  };

  template<>
  struct inverse_traits<ranges::minus> {
     using type = ranges::plus;
     constexpr type operator()() const noexcept { return type{}; }
  }
}
\end{itemdecl}

\rSec2[support.arithmetic.times]{Times}

\reviewernote{
   The term \tcode{multiplies} is --- in the author's opinion --- not the best name, and so the author
   would like to take the opportunity of rename this function object so that one can more
   naturally describe the computation.\\
   \\
   A potential alternative is \tcode{product}, this is the result of multiplication, not the
   operation itself (we'd need to rename \tcode{plus} to \tcode{sum}, etc., to facilitate that
   idea).
}

\begin{itemdecl}
namespace std::ranges {
  template<class T, class U>
  concept @\placeholder{multiplicable-with}@ = // \expos
    @\placeholder{summable-with}@<T, U> &&
    constructible_from<remove_cvref_t<T>, int> && // specifically T\{0\} and T\{1\}
    constructible_from<remove_cvref_t<U>, int> && // specifically U\{0\} and U\{1\}
    constructible_from<remove_cvref_t<common_type<T, U>>, int> &&
    common_reference_with<T, U> &&
    requires(T&& t, U&& u) {
      { std::forward<T>(t) * std::forward<T>(t) } -> common_with<T>;
      { std::forward<U>(u) * std::forward<U>(u) } -> common_with<U>;
      { std::forward<T>(t) * std::forward<U>(u) } -> common_with<T>;
      { std::forward<U>(u) * std::forward<T>(t) } -> common_with<U>;
      requires same_as<decltype(std::forward<T>(t) * std::forward<U>(u)),
                       decltype(std::forward<U>(u) * std::forward<T>(t))>;
    };
\end{itemdecl}
\begin{itemdescr}
   \pnum
   \tcode{T\{0\}} is equivalent to \tcode{T\{\}}, and \tcode{U\{0\}} is equivalent to \tcode{U\{\}}.

   \pnum
   The expressions
   \begin{itemize}
      \item \tcode{t * T\{\} == T\{\}},
      \item \tcode{u * U\{\} == U\{\}},
      \item \tcode{t * U\{\} == U\{\}}, and
      \item \tcode{u * T\{\} == T\{\}}
   \end{itemize}
   are all \tcode{true}.

   \pnum
   The expressions
   \begin{itemize}
      \item \tcode{t * T\{1\} == t},
      \item \tcode{T\{1\} * t == t},
      \item \tcode{u * U\{1\} == u},
      \item \tcode{U\{1\} * u == u},
      \item \tcode{u * T\{1\} == u},
      \item \tcode{T\{1\} * u == u},
      \item \tcode{t * U\{1\} == t}, and
      \item \tcode{U\{1\} * t == t}
   \end{itemize}
   are all \tcode{true}.
\end{itemdescr}

\begin{itemdecl}
  struct times {
    template<class T, @\placeholder{multiplicable-with}@<T> U>
    constexpr decltype(auto) operator()(T&& t, U&& u) const
    { return std::forward<T>(t) * std::forward<U>(u); }

    using is_transparent = std::true_type;
  };

  template<class T, class U>
  requires magma<times, T, U>
  struct left_identity<times> {
    constexpr common_type_t<T, U> operator()() const { return T{1}; }
  };

  template<class T, class U>
  requires magma<times, T, U>
  struct right_identity<times> {
    constexpr common_type_t<T, U> operator()() const { return U{1}; }
  };

  template<class T, class U>
  requires magma<times, T, U>
  struct left_zero<times> {
    constexpr common_type_t<T, U> operator()() const { return T{}; }
  };

  template<class T, class U>
  requires magma<times, T, U>
  struct right_zero<times> {
    constexpr common_type_t<T, U> operator()() const { return U{}; }
  };

  template<>
  struct inverse_traits<times> {
     using type = divided_by;
     constexpr type operator()() const noexcept { return type{}; }
  };
}
\end{itemdecl}

\rSec2[support.arithmetic.divided_by]{Divided by}

\reviewernote{
   The term \tcode{divides} clashes with the predicate \textit{divides}, which is used to indicate
   that a quotient is an integer. The author recommends renaming this operation to
   \tcode{divided_by}. A potential alternative is \tcode{quotient}, but this has the same issues as
   \tcode{product}.
}

\begin{itemdecl}
namespace std::ranges {
  template<class T, class U>
  concept @\placeholder{divisible-with}@ = // \expos
    @\placeholder{multiplicable-with}@<T, U> &&
    @\placeholder{subtractible-with}@<T, U> &&
    requires(T&& t, U&& u) {
      { std::forward<T>(t) / std::forward<T>(t) } -> common_with<T>;
      { std::forward<U>(u) / std::forward<U>(u) } -> common_with<U>;
      { std::forward<T>(t) / std::forward<U>(u) } -> common_with<T>;
      { std::forward<U>(u) / std::forward<T>(t) } -> common_with<U>;
      requires same_as<decltype(std::forward<T>(t) / std::forward<U>(u)),
                       decltype(std::forward<U>(u) / std::forward<T>(t))>;;
    };
\end{itemdecl}
\begin{itemdescr}
   \pnum
   Let \tcode{t1} and \tcode{t2} be objects of type \tcode{T}, and \tcode{u1} and \tcode{u2} be
   objects of type \tcode{U}. It is undefined for \tcode{t2 == T{0}} or \tcode{u2 == U{0}} to be
   \tcode{true} in all of the paragraphs below.

   \pnum
   The expressions
   \begin{itemize}
      \item \tcode{(t1 / t2) * t2 == t1},
      \item \tcode{(t1 * t2) / t2 == t1},
      \item \tcode{(u1 / u2) * u2 == u1},
      \item \tcode{(u1 * u2) / u2 == u1},
      \item \tcode{(t1 / u2) * u2 == t1},
      \item \tcode{(t1 * u2) / u2 == t1},
      \item \tcode{(u1 / t2) * t2 == u1}, and
      \item \tcode{(u1 * t2) / t2 == u1}
   \end{itemize}
   are all \tcode{true}.

   \pnum
   The expressions \tcode{T\{\} / t2 == T\{\}}, \tcode{U\{\} / u2 == U\{\}},
   \tcode{T\{\} / u2 == common_type_t<T, U>\{\}}, and \tcode{U\{\} / t2 == common_type_t<T, U>\{\}}
   are all \tcode{true}.

\end{itemdescr}

\begin{itemdecl}
  struct divided_by {
    template<class T, @\placeholder{divisible-with}@<T> U>
    constexpr decltype(auto) operator()(T&& t, U&& u) const
    { return std::forward<T>(t) / std::forward<U>(u); }
  };

  template<class T, class U>
  requires magma<divided_by, T, U>
  struct right_identity<divided_by> {
    constexpr common_type_t<T, U> operator()() const { return U{1}; }
  };

  template<>
  struct inverse_traits<divided_by> {
    using type = times;
    constexpr type operator()() const noexcept { return type{}; }
  };
}
\end{itemdecl}

\rSec2[support.arithmetic.modulus]{Modulus}

\reviewernote{
  An alternative name to \tcode{modulus} is \tcode{remainder}, which fits well with \tcode{sum},
  \tcode{difference}, \tcode{product}, and \tcode{quotient}.
}

\begin{itemdecl}
namespace std::ranges {
  template<class T, class Q>
  concept @\placeholder{modulo-with}@ = // \expos
    @\placeholder{divisible-with}@<T, Q> &&
    requires(T&& t, Q&& q) {
      { std::forward<T>(t) % std::forward<T>(t) } -> common_with<T>;
      { std::forward<Q>(q) % std::forward<Q>(q) } -> common_with<Q>;
      { std::forward<T>(t) % std::forward<Q>(q) } -> common_with<T>;
      { std::forward<Q>(q) % std::forward<T>(t) } -> common_with<Q>;
      requires same_as<decltype(std::forward<T>(t) % std::forward<Q>(q)),
                       decltype(std::forward<Q>(q) % std::forward<T>(t))>;
    };
\end{itemdecl}
\begin{itemdescr}
   \pnum
   Let \tcode{n} and \tcode{r} be objects of type \tcode{common_type_t<T, Q>}.

   \pnum
   The expression \tcode{t == q * n + r} is \tcode{true} if and only if \tcode{t \% q == r} is
   \tcode{true}.
\end{itemdescr}

\begin{itemdecl}
  struct modulus {
    template<class T, @\placeholder{modulo-with}@<T> U>
    constexpr decltype(auto) operator()(T&& t, U&& u) const
    { return std::forward<T>(t) % std::forward<U>(u); }
  };

  template<class T, class U>
  requires magma<modulus, T, U>
  struct left_zero<modulus> {
    constexpr common_type_t<T, U> operator()() const { return T{}; }
  };
}
\end{itemdecl}
