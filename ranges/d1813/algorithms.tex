%!TEX root = D1813.tex
\rSec0[algorithms]{Algorithms}

\begin{quote}
``We start with algorithms because it is algorithms we want to specify cleanly, precisely,
completely, and readably. If we can specify algorithms well, our concepts and the language
mechanisms we use to specify the concepts are adequate. If not, no amount of sophistication in
language mechanisms will help us.''
\begin{flushright}
\textemdash \textit{N3351, §2}
\end{flushright}
\end{quote}

\begin{quote}
``Generic Programming pro tip: Although Concepts are constraints on types, you don't find them by
looking at the types in your system. You find them by studying the algorithms.''
\begin{flushright}
\textemdash \textit{Eric Niebler, Twitter}
\end{flushright}
\end{quote}

\rSec1[algorithms.home]{Where do the numeric algorithms belong?}

The numeric algorithms have lived in \tcode{<numeric>} since the original implementation of STL, yet
many developers frequently question why they are not found in \tcode{<algorithm>}. With standard
module units on the horizon, it might seem pointless to discuss the validity of choosing to separate
the numeric algorithms from the rest of their kin. This section aims to provide some guidance for
when LEWG ultimately drafts the design for which entities reside in what module units.

\reviewernote{
   This section is pending reviewer input.
}

\rSec1[algorithms.sequenced]{Sequenced numeric algorithms}

The `sequenced numeric algorithms' are the algorithms found in \tcode{<numeric>}, introduced in the
STL; most of which made their way into \Cpp{}98. This family of algorithms performs computations in
a sequential manner, from left-to-right, or from the first element in the sequence to the last.
For some binary operation \tcode{bop}, and two expressions \tcode{x} and \tcode{y}, the expression
\tcode{bop(x, y)} need only be equality-preserving \cxxiref{concepts.equality}; the expression
\tcode{bop(x, y)} doesn't need to be \textit{associative}, nor does it need to be
\textit{commutative}. That is, \tcode{bop(x, bop(y, z))} is not required to return the same result
as \tcode{bop(bop(x, y), z)}, and \tcode{bop(x, y)} does not need to return the same result as
\tcode{bop(y, x)}.

\pnum
\begin{example}
   Addition is an associative operation: $1 + (2 + 3) = 6$ and $(1 + 2) + 3 = 6$ also. Subtraction
   is not an associative operation: $1 - (2 - 3) = 2$ and $(1 - 2) - 3 = -4$.
\end{example}

\pnum
\begin{example}
   Addition is an commutative operation: $1 + 2 = 3$ and $2 + 1 = 3$ also. Subtraction is not a
   commutative operation: $1 - 2 = -1$ and $2 - 1 = 1$.
\end{example}

\pnum
\begin{note}
   An operation does not need to be both associative and commutative; nor does it need to be
   neither.
   \begin{example}
      Matrix multiplication is associative, but not commutative\cite{wolfram-matrix-multiplication}.
   \end{example}
\end{note}

\rSec2[algorithms.accumulate]{Accumulate}

\tcode{accumulate} is an algorithm that performs a fold operation, or in other words, takes a
sequence of values and reduces them into a single value according to some operation. This is a
generalisation of a summation.  The algorithm -- modelled after what's currently in the
International Standard -- has a fairly straightforward declaration.

\begin{itemdecl}
template<input_iterator I, sentinel_for<I>, movable T, class Proj = identity,
         indirect_magma<const T*, projected<I, Proj>, T*> BOp = ranges::plus>
constexpr accumulate_result<I, T>
  accumulate(I first, S last, T init, BOp bop = {}, Proj proj = {});

template<input_range R, movable T, class Proj = identity,
         indirect_magma<const T*, projected<iterator_t<R>, Proj>, T*> BOp = ranges::plus>
constexpr accumulate_result<safe_iterator_t<R>, T>
  accumulate(R&& r, T init, BOp bop = {}, Proj proj = {});
\end{itemdecl}

\begin{itemdescr}
\pnum
A \textit{magma} is a binary operation \tcode{bop} over a set of elements $S$, where the result of
\tcode{bop(x, y)} is also in the set, or alternatively, \tcode{bop} is closed under
$S$\cite{wikipedia_magma}. Because different types may represent the same set of elements (e.g. all
of \tcode{int}, \tcode{long long}, and \tcode{double} can be used to represent a subset of
integers), \tcode{BOp} does not need to be a homogeneous binary operation. For equational reasoning
purposes, the types are expected to have a common type, and so \tcode{bop(0, vector{0})} does not
model a \tcode{magma}. Similarly, \tcode{bop(x, y)}, where \tcode{x} and \tcode{y} are possibly
different types is expected to share a type common to both \tcode{x} and \tcode{y}. The type of
\tcode{bop(x, y)} must be the same as the type of \tcode{bop(y, x)}. magma also requires \tcode{BOp}
to model \tcode{regular_invocable}. Finally, a magma is only concerned with closure: they do not
impose any requirements on associativity, nor on commutativity, so although the types of
\tcode{bop(x, y)} and \tcode{bop(y, x)} need to match, there is no requirement for their values to
match.
\end{itemdescr}

It might also be nice to use \tcode{accumulate} without an initial value, similarly to \Cpp{}17's
\tcode{std::reduce}. It would certainly be convenient to use \tcode{accumulate(r)} or even
\tcode{accumulate(r, ranges::times\{\})}, where \tcode{r} is an arbitrary range, and
\tcode{ranges::times} is a modernisation of \tcode{std::multiplies}. The former is fairly trivial
to do: we can default \tcode{init = T\{\}} and call it a day, just as \tcode{std::reduce} has, but
the author feels that this is lacking. An \textit{ideal} \tcode{init}-less \tcode{accumulate} should
permit the caller to specify a range, optionally an operation, and optionally a projection. This
requires great care, because \tcode{accumulate(r, times\{\})} when \tcode{init = T\{\}} would always
produce a single result: \tcode{T\{\}}. The reasons for why this is not desirable should be obvious.

By instead choosing an appropriate way to represent an operation's identity element,
\tcode{accumulate(r, bop, proj)} becomes a viable candidate to add to our overload set. An
\textit{identity element} $id$ is an element in aset $S$, where $x \cdot id$ is equivalent to $x$,
or $id \cdot x$ is equivalent to $x$. $x \cdot id$ is called a \textit{right-identity}, because $id$
is on the right-hand-side of $x$, and $id \cdot x$ is called a \textit{left-identity}. When $id$ is
both a left-identity and a right-identity, we call it a \textit{two-sided identity} (mathematicians
should note that \tcode{std::identity} is a function object\cxxiref{func.identity}).

\pnum
\begin{example}
   $0$ is the two-sided identity element for addition of real numbers: $x + 0 = x = 0 + x = x$.
\end{example}

\pnum
\begin{example}
   $1$ is the two-sided identity element for multiplication of real numbers: $1(x) = x = x(1) = x$.
\end{example}

With an interface that requires a two-sided identity, we can now declare our additions to the
\tcode{accumulate} overload set.

\begin{itemdecl}
template<input_iterator I, sentinel_for<I> S, class Proj = identity,
         indirect_monoid<projected<I, Proj>, projected<I, Proj>,
                         iter_value_t<projected<I, Proj>>*> BOp = ranges::plus>
  requires movable<iter_value_t<projected<I, Proj>>>
constexpr accumulate_result<I, iter_value_t<projected<I, Proj>>>
  accumulate(I first, S last, BOp bop = {}, Proj proj = {});

template<input_range R, class Proj = identity,
         indirect_monoid<projected<iterator_t<R>, Proj>,
                         projected<iterator_t<R>, Proj>,
                         iter_value_t<projected<iterator_t<R>, Proj>>* = ranges::plus>
  requires movable<iter_value_t<projected<iterator_t<R>, Proj>>>
constexpr accumulate_result<safe_iterator_t<R>, iter_value_t<projected<iterator_t<R>, Proj>>>
  accumulate(R&& r, BOp bop = {}, Proj proj = {});
\end{itemdecl}
\begin{itemdescr}
A \textit{monoid} is a twice-removed refinement over magma: it requires \tcode{BOp} be an
associative operation (this is a \textit{semigroup}\cite{wikipedia_semigroup}), and it requires that
\tcode{BOp} have a two-sided identity element\cite{wikipedia_monoid}. How this is achieved is
covered later, but it is a good idea to note now that the notion of identities are defined using a
new set of traits (numeric traits). This overload subset designates the return type to be the same
as the iterator's value type, so the requirement for \tcode{T} to be \tcode{movable} must be moved
appropriately.
\end{itemdescr}

\rSec2[algorithms.partial.sum]{Partial sum}

In mathematics, a partial sum is a summation of the first $N$ elements of a
sequence\cite{wolfram-partial-sum}.

$$S_N = \sum_{k = 0}^{N - 1} a_k$$

The \Cpp{} algorithm \tcode{partial_sum} is a generalisation of a partial sum, which writes the
$k$th evaluation of \tcode{accumulate} to an output range. The interface is extremely similar to
that of \tcode{accumulate}.

\begin{codeblock}
template<input_iterator I, sentinel_for<I> S1, weakly_incrementalbe O, sentinel_for<O> S2,
         class Proj = identity,
         indirect_magma<projected<I, Proj>, projected<I, Proj>, O> BOp = ranges::plus>
  requires indirectly_copyable_storable<I, O>
constexpr partial_sum_result<I, O>
  partial_sum(I first, S1 last, O result, S2 result_last, BOp bop = {}, Proj proj = {});

template<input_range R, range O, class Proj = identity,
         indirect_magma<projected<iterator_t<R>, Proj>,
                        projected<iterator_t<R>, Proj>,
                        iterator_t<O>> BOp = ranges::plus>
  requires indirectly_copyable_storable<iterator_t<I>, iterator_t<O>>
constexpr partial_sum_result<safe_iterator_t<R>, safe_iterator_t<O>>
  partial_sum(R&& r, O&& result, BOp bop = {}, Proj proj = {});
\end{codeblock}

Unlike \tcode{accumulate}, \tcode{partial_sum} doesn't require an initial value: it instead
designates \tcode{invoke(proj, *first)} as the initial value. \tcode{partial_sum} requires its
binary operation model a \tcode{magma} over its projected input range for the same reasons as
\tcode{accumulate}. The output of \tcode{partial_sum}'s value-type must be copyable, and movable via
a cache\cxxiref{alg.req.ind.copy}.

\reviewernote{
   The above paragraph is poorly worded. Input on how to rephrase is appreciated.
}

To minimise the likelidhood of writing to an beyond an output range that is smaller than the input
range, both overloads have been slightly altered to take two full ranges instead of a
range-and-a-half. The range-and-a-half overloads can be emulated using \tcode{unreachable_t}.

\rSec2[algorithms.adjacent.difference]{Adjacent difference}

\tcode{adjacent_difference} is a specialised transformation over adjacent elements in an input range
to compute the inverse of a \tcode{partial_sum} \iref{proof.adjacent.difference}. This yields some
interesting properties about \tcode{adjacent_difference}'s requirements, as shown below.

\begin{itemdecl}
template<input_iterator I, sentinel_for<I> S1, weakly_incrementalbe O, sentinel_for<O> S2,
         class Proj = identity,
         indirect_loop<projected<I, Proj>, projected<I, Proj>, O> BOp = ranges::minus>
  requires requires indirectly_copyable_storable<I, O>
constexpr adjacent_difference_result<I, O>
  adjacent_difference(I first, S1 last, O result, S2 result_last, BOp bop = {}, Proj proj = {});

template<input_range R, range O, class Proj = identity,
         indirect_loop<projected<iterator_t<R>, Proj>,
                       projected<iterator_t<R>, Proj>,
                       iterator_t<O>> BOp = ranges::minus>
  requires indirectly_copyable_storable<iterator_t<I>, iterator_t<O>>
constexpr adjacent_difference_result<safe_iterator_t<R>, safe_iterator_t<O>>
  adjacent_difference(R&& r, O&& result, BOp bop = {}, Proj proj = {});
\end{itemdecl}
\begin{itemdescr}
   \pnum
   A \textit{loop} is another twice-removed refinement over a magma. Specifically, it requires that
   the binary operation have an inverse operation (this is a
   \textit{quasigroup}\cite{wikipedia_quasigroup}), and a two-sided identity. It is necessary for
   \tcode{adjacent_difference} to require a loop, so that we can guarantee that it is the inverse
   algorithm of \tcode{partial_sum}.

   It's important to note that despite appearing to have similar use-cases, both the the interface
   and implementation for \tcode{adjacent_difference} are distinct from
   \tcode{transform}\cxxiref{alg.transform}:

   \begin{description}
      \item[Varying interface]
         \tcode{adjacent_difference} requires that its binary operation model
         \tcode{regular_invocable}\cxxiref{concept.regularinvocable}, while \tcode{transform} only
         requires its binary operation model
         \tcode{invocable}\cxxiref{concept.invocable}\cxxiref{indirectcallable.indirectinvocable}.
      \item[Varying implementation]
         \tcode{transform} applies its operands in the order of left-to-right, à la
         \tcode{op(*first1, *first2)}, while \tcode{adjacent_difference} applies its operands in the
         \textit{opposite} order, à la \tcode{bop(prev, *first)}.
   \end{description}
\end{itemdescr}

\reviewernote{
   Despite \textit{loop} being the technical term for this algebraic structure, the author does not
   encourage the using the name \tcode{loop} \textit{directly}, due to the likelihood of it being
   confused with the computer science term `loop'. See \ref{support.concepts.loop} for possible
   alternative (ugly) names, and \ref{support.concepts.group} for a possible (and preferred)
   redesign.
}

\rSec2[algorithms.inner.product]{Inner Product}

\tcode{inner_product} generalises an algebraic inner product into a map-reduce operation.

\begin{itemdecl}
template<input_iterator I1, sentinel_for<I1> S1, input_iterator I2, sentinel_for<I2> S2,
         movable T, class Proj1 = identity, class Proj2 = identity,
         class BOp1 = ranges::plus, class BOp2 = ranges::times>
  requires indirect_weak_magmaring<BOp1, BOp2, const T*,
                                   projected<I1, Proj1>, projected<I2, Proj2>, T*>
constexpr inner_product_result<I1, I2, T>
 inner_product(I1 first1, S1 last1, I2 first2, S2 last2, T init,
               BOp1 bop1 = {}, BOp2 bop2 = {}, Proj1 proj1 = {}, Proj2 proj2 = {});

template<input_range R1, input_range R2, class BOp1 = ranges::plus, class BOp2 = ranges::times,
         movable T, class Proj1 = identity, class Proj2 = identity>
  requires indirect_weak_magmaring<BOp1, BOp2, const T*,
                                   projected<iterator_t<R1>, Proj1>,
                                   projected<iterator_t<R2>, Proj2>, T*>
constexpr inner_product_result<safe_iterator_t<R1>, safe_iterator_t<R2>, T>
  inner_product(R1&& r1, R2&& r2, T init, BOp1 bop1 = {}, BOp2 bop2 = {},
                Proj1 proj1 = {}, Proj2 proj2 = {});
\end{itemdecl}
\begin{itemdescr}
   \pnum
   A \tcode{weak_magmaring} is an extreme generalisation of the well-known semiring algebraic
   structure that establishes a relationship between two magmas. Specifically, it describes that a
   magma \tcode{BOp2} is \textit{distributive} over \tcode{BOp1}. Given three objects of possibly
   distinct -- but related -- types, \tcode{x}, \tcode{y}, and \tcode{z}, the expression
   \tcode{bop2(x, bop1(y, z))} is equivalent to \tcode{bop1(bop2(x, y), bop2(x, z))}.

   \pnum
   \begin{example}
      Multiplication is distributive over addition: $x(y + z) = xy + yz$.
   \end{example}

   \pnum
   Mathematicians note that weak-magmaring is a generalisation of a near-semiring, named by the
   author, to fit the requirements. The author asked around on StackExchange\cite{stack-exchange-wm}
   before naming this algebraic structure, but it seems that the structure is too general to be of
   interest outside of this use-case. The naming decision stems from that fact that a near-semiring
   weakens $(S, \cdot)$ from a monoid to a semigroup, and a weak-magmaring weakens $(S, \cdot)$ from
   a semiring to a magma. A more appropriate name might exist: near-semirings still require $(S, +)$
   to model a monoid, but a near-magma weakens this requirement to a magma as well.

   \pnum
   Similarly to \tcode{adjacent_difference}, \tcode{inner_product} is not quite the same as
   \Cpp{}17's \tcode{transform_reduce}, which is expected to be far more permissive with its
   operations.
\end{itemdescr}

Care has been taken to ensure that \tcode{inner_product} is not over-constraining, and that only
types that directly interact are required to have a common type. This means the following code
doesn't meet the requirements for \tcode{inner_product}.

\begin{codeblock}
auto words_to_ints = [](string_view const word) -> int {
   // ...
};
auto const data1 = vector{"one"s, "two"s, "three"s, "four"s, "five"s};
auto const data2 = vector{"six"s, "seven"s, "eigth"s, "nine"s, "ten"s};
return inner_product(data1, data2, 0, ranges::plus{}, words_to_ints);
// error: words_to_ints doesn't model magma<string, string>, since
//        common<invoke_result_t<words_to_ints, string, string>, int> is false.
\end{codeblock}

A user that wants to perform this operation should instead use the following:
\begin{codeblock}
auto as_words = view::transform(words_to_ints);
return accumulate(view::zip_with(data1 | as_words, data2 | as_words, ranges::times{}));
\end{codeblock}
\reviewernote{
   The author painfully is aware that \tcode{zip_with_view} is yet to be standardised: this use-case
   exasperates the need for such a library feature.
}

Similarly to \tcode{accumulate}, by refining our requirements, it's possible to eliminate the need
for an initial value, thereby making this possible:

\begin{codeblock}
auto ints = view::iota(0);
auto slice = [](auto const drop, auto const take) {
   return view::drop(drop) | view::take(take);
};
return inner_product(ints | slice(100, 10), ints | slice(10, 10));
\end{codeblock}

\begin{codeblock}
template<input_iterator I1, sentinel_for<I1> S1, input_iterator I2, sentinel_for<I2> S2,
         class BOp1 = ranges::plus, class BOp2 = ranges::times,
         class Proj1 = identity, class Proj2 = identity>
requires indirect_near_semiring<BOp1, BOp2,
                                const iter_value_t<projected<I1, Proj1>>*,
                                projected<I1, Proj1>,
                                projected<I2, Proj2>,
                                iter_value_t<projected<I1, Proj1>>*>
constexpr inner_product_result<I1, I2, iter_value_t<projected<I1, Proj1>>>
inner_product(I1 first1, S1 last1, I2 first2, S2 last2, BOp1 bop1 = {}, BOp2 bop2 = {},
              Proj1 proj1 = {}, Proj2 proj2 = {});

template<input_range R1, input_range R2, class Proj1 = identity, class Proj2 = identity,
         class BOp1 = ranges::plus, class BOp2 = ranges::times>
requires indirect_near_semiring<BOp1, BOp2,
                                const iter_value_t<projected<iterator_t<R1>, Proj1>>*,
                                projected<iterator_t<R1>, Proj1>,
                                projected<iterator_t<R2>, Proj2>,
                                iter_value_t<projected<iterator_t<R1>, Proj1>>*>
constexpr inner_product_result<safe_iterator_t<R1>, safe_iterator_t<R2>,
                               iter_value_t<projected<iterator_t<R1>, Proj1>>>
inner_product(R1&& r1, R2&& r2, BOp1 bop1 = {}, BOp2 bop2 = {},
              Proj1 proj1 = {}, Proj2 proj2 = {});
\end{codeblock}

A \textit{near-semiring} is a refinement of a weak-magmaring, and naturally arises from studying
functions on monoids\cite{wikipedia_near_semiring}. A near-semiring requires that \tcode{BOp1} model
a monoid, and that \tcode{BOp2} model a semigroup. As a near-semiring refines a weak-magmaring, it
subsumes the distributive property. It also introduces the notion of an
\textit{annhiliating element}\cite{wikipedia_absorb}. In mathematics, an annihilating element is a
special element in a set for certain operations, such that when applied with any other element in
the set, the result of the operation is the annihliating element. It is the complete opposite of an
identity element.

\begin{example}
   For example, scalar multiplication's annihilating element is $0$: $0x = 0$ and $x0 = 0$.
\end{example}

Semigroup theory refers to annihilating elements as the \textit{zero element}, as there is only one
notion of zero.

\reviewernote{
   While a zero element is not strictly a necessity for \tcode{inner_product}, it is a fundamental
   property of a near-semiring, and so it has been included in the requirements for a
   \tcode{near_semiring}.
}

\rSec2[algorithms.iota]{Iota}

\reviewernote{
   As the \Cpp{}20 WP contains \tcode{iota_view}, it is unclear to the author whether or not there
   is a place for an algorithm \tcode{iota}. This subsection will be filled out, either in favour or
   against, after receiving guidance.
}

\rSec2[algorithms.power]{Power}

\reviewernote{
   While reviewing the history of the original STL implementation, the author noted that there
   existed an extension algorithm called \tcode{power}. The current revision of this document does
   not explore this algorithm, but a future revision may.
}

\rSec1[algorithms.unsequenced]{Unsequenced numeric algorithms}

The `unsequenced numeric algorithms' are the \tcode{<numeric>} algorithms introduced in \Cpp{}17.
These are a further generalisation of the sequenced numeric algorithms, and may perform computations
out-of-order. As such, in order to guarantee equality-preservation, these algorithms will require
their operations be both associative and commutative.

\rSec2[algorithms.reduce]{Reduce}

\tcode{reduce} is the unsequenced counterpart to \tcode{accumulate}. Its declaration is fairly
similar to that of \tcode{accumulate}, except for the refinements introduced by this section.

\begin{codeblock}
template<input_iterator I, sentinel_for<I>, movable T, class Proj = identity,
         indirect_commutative_semigroup<const T*, projected<I, Proj>, T*> BOp = ranges::plus>
constexpr reduce_result<I, T>
  reduce(I first, S last, T init, BOp bop = {}, Proj proj = {});

template<input_range R, movable T, class Proj = identity,
         indirect_commutative_semigroup<const T*,
                                        projected<iterator_t<R>, Proj>, T*> BOp = ranges::plus>
constexpr reduce_result<safe_iterator_t<R>, T>
  reduce(R&& r, T init, BOp bop = {}, Proj proj = {});

template<input_iterator I, sentinel_for<I> S, class Proj = identity,
         indirect_commutative_monoid<projected<I, Proj>, projected<I, Proj>,
                                     iter_value_t<projected<I, Proj>>*> BOp = ranges::plus>
requires movable<iter_value_t<projected<I, Proj>>>
constexpr reduce_result<I, iter_value_t<projected<I, Proj>>>
  reduce(I first, S last, BOp bop = {}, Proj proj = {});

template<input_range R, class Proj = identity,
         indirect_commutative_monoid<projected<iterator_t<R>, Proj>,
                                     projected<iterator_t<R>, Proj>,
                                     iter_value_t<projected<iterator_t<R>, Proj>>* = ranges::plus>
requires movable<iter_value_t<projected<iterator_t<R>, Proj>>>
constexpr reduce_result<safe_iterator_t<R>, iter_value_t<projected<iterator_t<R>, Proj>>>
  reduce(R&& r, BOp bop = {}, Proj proj = {});
\end{codeblock}

\tcode{commutative_semigroup} and \tcode{commutative_monoid} respectively refine \tcode{semigroup}
and \tcode{monoid} so that \tcode{BOp} is a commutative operation. This is achieved by introducing a
\tcode{commutative_operation} concept, which requires that for two distinct values \tcode{x} and
\tcode{y}, \tcode{bop(x, y)} is has same result as \tcode{bop(y, x)}.

\reviewernote{
   This document does not yet define the concepts \tcode{commutative_semigroup} and co, but one can
   `imagine' them being equivalent to \tcode{semigroup<T, U> \&\& commutative_operation<T, U>}, etc.
}

\rSec2[algorithms.inclusive.scan]{Inclusive scan}

This revision does not explore the requirements for \tcode{inclusive_scan}.

\rSec2[algorithms.exclusive.scan]{Exclusive scan}

This revision does not explore the requirements for \tcode{exclusive_scan}.

\rSec2[algorithms.transform.reduce]{Transform reduce}

This revision does not explore the requirements for \tcode{transform_reduce}.
