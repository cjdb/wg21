\rSec0[intro]{Introduction}

\begin{quote}
``Every time someone asks why we didn’t cover \tcode{<numeric>} and \tcode{<memory>} algorithms: We
thought 187 pages of TS was enough.''
\begin{flushright}
\textemdash \textit{Casey Carter}
\end{flushright}
\end{quote}

\rSec1[intro.motivation]{Motivation}

N3351\cite{N3351} served as the basis for the Ranges TS\cite{N4685}, which was merged into the
\Cpp{}20 Working Paper\cite{P0898}\cite{P0896}. N3351 focused on defining concepts for the standard
library, which is achieved by looking at the use-cases that concepts are designed for: generic
algorithms. Specifically, N3351 looked at pinning down the concepts relevant to the algorithms found
in \tcode{<algorithm>} after \Cpp{}11. All known bodies of work from N4128\cite{N4128} through to
P0896 and P0898 -- with the exception of P1033\cite{P1033} -- have continued to focus on studying
and refining the contents of \tcode{<algorithm>}. P1033 takes the extremely low-hanging fruit and
adds the uninitialised-memory algorithms from \tcode{<memory>} to the mix. To the author's best
knowledge, all that's left to be added are possibly a few algorithms introduced in \Cpp{}20, and all
of the algorithms in \tcode{<numeric>}.

The numeric algorithms weren't abandoned or forgotten: given the limited resources, there simply
wasn't enough time to study all of the algorithms in \tcode{<algorithm>} and \tcode{<numeric>}, and
also introduce the basis for range adaptors in \Cpp{}20. Now that we're moving into the \Cpp{}23
design space, we should start reviewing the numeric algorithms in the same light as N3351 considered
the \tcode{<algorithm>} algorithms.

A complete design is not as simple as taking the concepts introduced in P0896, slapping them on the
numeric algorithms, and calling it a day. These algorithms have different requirements to those in
\tcode{<algorithm>}, and P1813 takes aim at what those might look like. The current revision chooses
to focus on only those algorithms introduced in \Cpp{}98 and \tcode{reduce}; the remaining \Cpp{}17
numeric algorithms are left to a subsequent revision.

\rSec1[intro.design.ideals]{Design ideals}

The following section has been lifted almost completely verbatim from N3351. This serves as a
reminder that the design ideals have not really changed since N3351's publication in 2012.
Non-editorial changes are represented by showing \changed{what is present in N3351}{and what is
present in P1813}.
\begin{enumerate}
   \item The concepts for the STL must be mathematically and logically sound. By this, we mean to
         emphasise the fact that we should be able to reason about properties of programs (e.g.
         correctness) with respect to the semantics of the language and the types used in those
         programs.
   \item The concepts used should express general ideas in the application domain (hence the name
         `concepts') rather than mere programming language artifacts. Thinking about concepts as a
         yet another `contract' language can lead to partially formed ideas. Contracts force
         programmers to think about requirements on individual functions or interfaces, whereas
         concepts should represent fully formed abstractions.
   \item The concepts should specify both syntactic and semantic requirements (``concepts are all
         about semantics'' -- Alex Stepanov). A concept without semantics only partially specifies
         an interface and cannot be reasoned about; the absence of semantics is the opposite of
         soundness (``it is insanity'' -- Alex Stepanov).
   \item Symbols and identifiers should be associated with their conventional meanings. Overloads
         should have well defined semantics and not change the usual meaning of the symbol or name.
   \item The concepts as used to specify algorithms should be terse and readable. An algorithm's
         requirements must not restate the syntax of its implementation.
   \item The number of concepts used should be low, in order to make them easier to understand and
         remember.
   \item An algorithm's requirements must not inhibit the use of very common code patterns in its
         implementation.
   \item An algorithm should not contain requirements for syntax that it does not use, thereby
         unnecessarily limiting its generality.
   \item The STL with concepts should be compatible with \changed{\Cpp{}11}{\Cpp{}20}, except where
         that compatibility would imply a serious violation of one of the first two aims.
\end{enumerate}

The following quote has also been extracted from N3351.

``Every generic library design must choose the style in which it describes template requirements.
The ways in which requirements are specified has a direct impact on the design of the concepts
used to express them, and (as always) there are direct consequences of that choice. For example,
we could choose to state template requirements in terms of the exact syntax requirements of the
template. This leads to concept designs that have large numbers of small yntactic predicates (e.g.
\tcode{HasPlus}, \tcode{HasComma}, etc.). The benefit of this style of constraint is that templates
are more broadly adaptable: there are potentially more conforming types with which the template will
interoperate. On the downside, exact requirements tend to be more verbose, decreasing the
likelihood that the intended abstraction will be adequately communicated to the library’s users.
The C++0x design is, in many aspects, a product of this style.

On the other end of the spectrum, we could choose to express requirements in terms of the required
abstraction instead of the required syntax. This approach can lead to (far) fewer concepts in the
library design because related syntactic requirements are grouped to create coherent, meaningful
abstractions. Requirements can also be expressed more tersely, needing fewer concepts to express a
set of requirements that describe how types are used in an algorithm. The use of abstract concepts
also allows an algorithm to have more conforming implementations, giving a library author an
opportunity to modify (i.e. maintain) a template’s implementation without impacting its
requirements. The obvious downside to this style is that it over-constrains templates; there may
be types that conform to a minimal set of operations used by a template, but not the full set of
operations required by the concept. The concepts presented in \textit{Elements of Programming}
approach this end of the spectrum.''

Similarly to N3351, P1813 aims to hold itself in-between these two extremes.

\rSec1[intro.organisation]{Organisation}

Similarly to N3351, P1813 is broken into a section for declaring algorithms with concept
requirements, and a section for defining concepts. This document is intended to be read in
sequentially, with many sections depending on exposition from previous sections.

\rSec1[intro.assumed.knowledge]{Assumed knowledge}

The following sections assume familiarity with the concepts library \cxxiref{concepts}, the
iterator concepts \cxxiref{iterator.concepts}, the indirect callable requirements
\cxxiref{indirectcallable}, the common algorithm requirements \cxxiref{alg.req}, the range
requirements \cxxiref{range.req}, and the way in which algorithms are specified in namespace
\tcode{std::ranges} \cxxiref{algorithms}.

Readers should consult \textit{Design of concept libraries for \Cpp{}}\cite{concept-design} prior to
reading the remainder of P1813. Readers are also encouraged to consult \textit{Elements of
Programming}\cite{EoP} and N3351 as necessary.
