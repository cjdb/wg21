\rSec0[motivation]{Motivation}

The ability to use algorithms over an evenly-spaced subset of a range has been missed in the STL for
a quarter of a century. This is, in part, due to the complexity required to use an iterator that can
safely describe such a range. It also means that the following examples cannot be transformed from
raw loops into algorithms, due to a lacking iterator.

\begin{codeblock}
for (auto i = 0; i < ssize(v); i += 2) {
  v[i] = 42; // fill
}

for (auto i = 0; i < ssize(v); i += 3) {
  v[i] = f(); // transform
}

for (auto i = 0; i < ssize(v); i += 3) {
  for (auto j = i; j < ssize(v); i += 3) {
    if (v[j] < v[i]) {
      ranges::swap(v[i], v[j]); // selection sort, but hopefully the idea is conveyed
    }
  }
}
\end{codeblock}

Boost introduced a range adaptor called \tcode{strided}, and range-v3's equivalent is
\tcode{stride\_view}, both of which make striding far easier than when using iterators:

\begin{codeblock}
ranges::fill(v | views::stride(2), 42);

auto strided_v = v | views::stride(3);
ranges::transform(strided_v, ranges::begin(strided_v) f);

ranges::stable_sort(strided_v); // order restored!
\end{codeblock}

Given that there's no way to compose a strided range adaptor in C++20, this should be one of the
earliest range adaptors put into C++23.

\subsection{Risk of not having \tcode{stride\_view}}

Although it isn't possible to compose \tcode{stride\_view} in C++20, someone inexperienced with the
ranges design space might mistake \tcode{filter\_view} as a suitable way to ``compose''
\tcode{stride\_view}:

\begin{codeblock}
auto bad_stride = [](auto const step) {
  return views::filter([n = 0, step](auto&&) mutable {
    return n++ % step == 0;
  });
};
\end{codeblock}

This implementation is broken for two reasons:

\begin{enumerate}
  \item \tcode{filter\_view} expects a \tcode{predicate} as its input, but the lambda we have
        provided does not model \tcode{predicate} (a call to \tcode{invoke} on a \tcode{predicate}
        mustn't modify the function object, yet we clearly are).
  \item The lambda provided doesn't account for moving backward, so despite \textit{satisfying}
        \tcode{bidirectional_iterator}, it does not model the concept.
\end{enumerate}

For these reasons, the author regrets not proposing this in the C++20 design space.
